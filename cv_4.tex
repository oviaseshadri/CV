%%%%%%%%%%%%%%%%%%%%%%%%%%%%%%%%%%%%%%%%%
% Medium Length Professional CV
% LaTeX Template
% Version 2.0 (8/5/13)
%
% This template has been downloaded from:
% http://www.LaTeXTemplates.com
%
% Original author:
% Trey Hunner (http://www.treyhunner.com/)
%
% Important note:
% This template requires the resume.cls file to be in the same directory as the
% .tex file. The resume.cls file provides the resume style used for structuring the
% document.
%
%%%%%%%%%%%%%%%%%%%%%%%%%%%%%%%%%%%%%%%%%

%----------------------------------------------------------------------------------------
%	PACKAGES AND OTHER DOCUMENT CONFIGURATIONS
%----------------------------------------------------------------------------------------

\documentclass{resume} % Use the custom resume.cls style

\usepackage[left=0.75in,top=0.6in,right=0.75in,bottom=0.6in]{geometry} % Document margins
\usepackage[pdftex]{hyperref}
\newcommand{\tab}[1]{\hspace{.2667\textwidth}\rlap{#1}}
\newcommand{\itab}[1]{\hspace{0em}\rlap{#1}}
\name{Curriculum Vitae of Ovia Seshadri} % Your name
\address{Ph.D. Scholar, Department of Computer Science, IIT Delhi, Huaz Khas, Delhi, India-110016} % Your address
%\address{123 Pleasant Lane \\ City, State 12345} % Your secondary addess (optional)
\address{(+91)~88262~88396 \\ oviaseshadri@gmail.com}  % Your phone number and email

\begin{document}

%----------------------------------------------------------------------------------------
%	EDUCATION SECTION
%----------------------------------------------------------------------------------------

\begin{rSection}{Education}
\textbf{Doctorate of Philosophy (Ph.D.) in Computer Science} \hfill {\em July 2015 - Present$^\mathsection$} 
\\\textit{Indian Institute of Technology(IIT), Delhi}  \hfill { CGPA: 8.25/10}
\\ Advised by Dr Vinay Ribeiro (IITB) and Dr Subodh Sharma (IITD)
\\ Department of Computer Science and Engineering\\ 
$\mathsection$ Synopsis completed and Thesis submitted

{\bf Thesis Summary -} \textit{Securely Improving Performance in PoW Blockchains using Links and Anchors:}\\ 
% PoW blockchains add blocks, and consequently the chain weight, randomly. The blocks added also have a significant network delay owing to their large size. Large delay combined with randomness causes forks that are responsible for many security problems. One can reduce fork occurrences by designing a system with large block intervals and size but this design compromises performance aspects such as confirmation time guarantees. The trade-off between security and performance in PoW blockchain is a well discussed topic in the literature. 
We reduce the trade-off between security and performance in PoW blockchains through our novel concepts of \textit{Links} and \textit{Anchors}. They are small, fast and frequent structures that can be incorporated on any new or existing PoW blockchains. They help reduce the confirmation time of their underlying blockchain while preserving its consistency security guarantees. 
% Included Publications: \textit{(1) “Securely Boosting Chain Growth and Confirmation Speed in PoW Blockchains with Links”, Ovia Seshadri, Vinay J, Ribeiro and Aditya Kumar, IEEE Blockchain 2021; (2) “Securely Improving Stability and Performance of PoW Blockchains using Anchors”, Ovia Seshadri, Vinay J, Ribeiro and Shadab Zafar, COMSNETS 2022.}
%Minor in Linguistics \smallskip \\
%Member of Eta Kappa Nu \\
%Member of Upsilon Pi Epsilon \\



% \vspace{0.1cm}
\textbf{M.S. Software Engineering} (5 year B.Tech. + M.S. Integrated course) \hfill {\em July 2008 - May 2013} 
\\\textit{VIT University, Vellore} \hfill { CGPA: $9.16/10^\dagger$}\\
% \\ School of Information Technology and Engineering \\
$\dagger$ Graduated in top 5\% in a batch of 300+

% {\bf Higher Secondary}\\
% \textit{Vidyaniketan High School, Bangalore} \hfill {\em July 2006 - March 2008; 12th ISC; 81\%} \\
% % \\ 12th ISC Board Examinations, New Delhi \hfill { $81\%^\ddagger$}
% \textit{Bishop Cottons Girls High School, Bangalore}\hfill {\em May 2006; 10th ICSE; 87\%} 
% % \\ 10th ICSE Board Examinations, New Delhi \hfill { $87\%^\star$}

% {\bf Vidyaniketan High School, Bangalore} \hfill {\em July 2006 - March 2008} 
% \\ 12th ISC Board Examinations, New Delhi \hfill { $81\%^\ddagger$}
% \\ $\ddagger$ First Class with Distinction; Scored top of the batch in Maths and English

% {\bf Bishop Cottons Girls High School, Bangalore} \hfill {\em May 2006} 
% \\ 10th ICSE Board Examinations, New Delhi \hfill { $87\%^\star$}
% \\ $\star$ First Class with Distinction; Scored top of the batch in Maths



\end{rSection}

%----------------------------------------------------------------------------------------
%	Research interest SECTION
%----------------------------------------------------------------------------------------

\begin{rSection}{Research Interests}

Blockchain, Network Security, Distributed Systems, Databases

\end{rSection}

\begin{rSection}{Research Publications}
\begin{itemize}
	\item “Securely Improving Stability and Performance of PoW Blockchains using Anchors”, \textbf{Ovia Seshadri}, Vinay J, Ribeiro and Shadab Zafar, COMSNETS 2022.
% 	{\bf Paper Summary -} Proof-of-work (PoW) consensus generates blocks at random time instants, and consequently, adds weight to the blockchain at these same instants. This unsteady increase in chain weight over time is the root cause of many security and performance problems in the form of forks and attacks due to forks. This work reduces confirmation time, resolves forks faster and prevents forks in new and existing blockchain systems.\\
% 	\textit{URL - https://ieeexplore.ieee.org/abstract/document/9668572}
	


\item “Securely Boosting Chain Growth and Confirmation Speed in PoW Blockchains”, \textbf{Ovia Seshadri}, Vinay J, Ribeiro and Aditya Kumar, IEEE Blockchain 2021.
% {\bf Paper Summary -} A major short coming of PoW blockchains is their inability to scale to low confirmation times required for typical micro payments. The main reason for this is the unsteady growth in chain weight. This work proposes a simple solution where chain growth is maintained in a steady manner and linear structure of the blockchain is maintained. We also lower confirmation times while maintaining security guarantees. Links are a better alternative to solutions that try to improve confirmation times by reducing block size and interval.\\
% \textit{URL - https://ieeexplore.ieee.org/abstract/document/9680576}

% {\bf Paper Summary -} \textit{Securely Improving Performance in PoW Blockchains using Links and Anchors:}\\ PoW blockchains add blocks, and consequently the chain weight, randomly. The blocks added also have a significant network delay owing to their large size. Large delay combined with randomness causes forks that are responsible for many security problems. One can reduce fork occurrences by designing a system with large block intervals and size but this design compromises performance aspects such as confirmation time guarantees. The trade-off between security and performance in PoW blockchain is a well discussed topic in the literature. In this thesis, we aim to reduce the conflict between security and performance through our novel concepts of \textit{Links} and \textit{Anchors}. They are small, fast and frequent structures that can be incorporated on any new or existing PoW blockchains. They help reduce the confirmation time of its underlying blockchain while preserving its consistency security guarantees. 
% 	URL - https://ieeexplore.ieee.org/abstract/document/9668572?casa\_token=0okTYT9EGYEAAAAA:69s\_rKf-Uh3hrgyfbUN4iHcN7aL63DZTrO5C\_igE9ni4qQfj8J48uA8W9J8mYTasj48UDzO\_T2E

\end{itemize}
\end{rSection}


\begin{rSection}{Patents}
\begin{itemize}
\item “Method in blockchain systems for fast stabilization and increased responsiveness using Anchors”, Vinay J. Ribeiro and Ovia Seshadri, \\
	Patent submitted in India (2019) (Application Number - 201911004921) \\
	Patent submitted in USA (2021) (US20220108313A1; US17/428,304)\\
	International PCT (2019) - WO2020161530A1\\
	\textit{URL - https://patents.google.com/patent/US20220108313A1}
	
	
	\item “Method in blockchain systems for fast stabilization and increased responsiveness using Links”, Vinay J. Ribeiro and Ovia Seshadri, \\
	Patent submitted in India (2020) (Application Number - 201911023814)\\
	International PCT (2020) - WO2020254923A1\\
	\textit{URL - https://patents.google.com/patent/WO2020254923A1}
\end{itemize}
\end{rSection}

\begin{rSection}{Selected Talks}
\begin{itemize}
\item Invited to present a talk ``Securely improving performance of PoW blockchains using anchors" at Workshop on Blockchains and Networking at ACM Sigmetrics 2022 in Mumbai in June 2022.

\item Invited to present a talk ``Mechanisms for improved security and performance of PoW Blockchains" at the National Research Evaluation Workshop of the Vishvesvaraya PhD scheme for electronics and IT by the Ministry of Electronics and Information Technology (MeitY), Government of India in Chandigarh in July 2019.

\item Invited to present a talk on ``Near real time consensus using Hashgraph in IoT systems" at the 5th International Workshop on Cyber security in Kyushu University, Fukuoka, Japan in July 2017.
\end{itemize}
\end{rSection}


%----------------------------------------------------------------------------------------
%	WORK EXPERIENCE SECTION
%----------------------------------------------------------------------------------------

\begin{rSection}{Work Experience}

% \begin{rSubsection}{IIT Delhi}{July 2015 - Present}{Doctoral Researcher}{}
% \item Ongoing Projects with Dr Vinay Ribeiro, IIT Bombay. Details in research project section. 
% \item Interned at IBM IRL. Details in Internships and research project sections.
% \item Worked on a sponsored Indo-japanese project. Details in Research projects section.
% \item Teaching Assistant for Data Mining, DBMS, Computer Networks, Cryptography and Network Security mostly under Dr. Bijendra Nath Jain, IIIT Delhi.
% \item Co-ordinating a research reading group called "SecVisor" at IIT Delhi. http://www.cse.iitd.ernet.in/~kumarsandeep/secvisor/
% \end{rSubsection}


%------------------------------------------------

\begin{rSubsection}{Aris Global Software Pvt. Ltd.}{July 2013 - June 2015}
{Principal Software Engineer}{}
\item Promoted to Principal Software Engineer from Senior Software Engineer from Dec 2014 based on performance in the project.
\item Worked on Project "agHub" which is a DWH solution for safety systems. It is a data mart which efficiently stores safety data for reporting purposes.
\item Was designated the role of data modeler and developed a dimensional model of the DWH. Involved in the development of ETL jobs to trafer data from RDBMS to DWH. Involved in lending support to reporting teams working on BO and COGNOS.
\item Involved with customers on discussions related to requirement analysis and issue resolution. Provided support to the testing and QC team in understanding the project setup and project overview.

\end{rSubsection}

%------------------------------------------------

% \begin{rSubsection}{VIT University}{July 2008 - May 2013}{}{}
% \item Graduated in top 5\% in a Batch of 300+
% \item Served as Director of documentation for the Computer Society of India (CSI) Vellore chapter.
% \item Organised several workshops including "National Workshop on Cryptography" in August 2012
% \item A Cognizant Certified Student
% \item Completed project titled "WEBEEPS-Revolutionizing the concept of collaboration through web based communication"
% \item Completed project titled "AgCarbon - A platform of highly scalable and reusable components for Life Sciences Industry"
% \end{rSubsection}

\end{rSection}





\begin{rSection}{Internships}
\begin{itemize}
\item IBM IRL, Bangalore from July to October 2017, on project titled “Blockchain like
Relational Database” under Dr Praveen Jayachandran. Details in Research project section.
\item Aris global Pvt. Ltd., Bangalore from January to June 2013 on Project titled "AgCarbon - A platform of highly scalable and reusable components for Life Sciences Industry" under the guidance of Anish Anand.
\item Monkey Creative Labs Pvt. Ltd., Chennai from May to December 2011, on Project titled "WEBEEPS-Revolutionizing the concept of collaboration through web based communication" under the guidance of Sharadha Ramakrishnan.
\end{itemize}
\end{rSection}


\begin{rSection}{Peer Reviews}
\begin{itemize}
\item Reviewer for Transactions on Management Information Systems(TMIS) and Conference on COMmunication Systems & NETworkS (COMSNETS)
\end{itemize}
\end{rSection}

\begin{rSection}{Research Projects}
\begin{itemize}

\item\textbf{Anchors for stability, Security and performance of PoW Blockchains}\\
Proof-of-work (PoW) consensus generates blocks at random time instants, and consequently, adds weight to the blockchain at these same instants. This unsteady increase in chain weight over time is the root cause of many security and performance problems in the form of forks. This work tries to prevent forks and reduce chances of selfish mining and double spend attacks in existing blockchain systems. This work has Indian and USA patents and an International PCT filing. It is published in IEEE Blockchain 2021.\\
	\textit{URL - https://ieeexplore.ieee.org/abstract/document/9668572}\\
Advisor: Dr Vinay Ribeiro, IIT Bombay

\item\textbf{Links: Making PoW Blockchains robust via steady chain growth}\\
A major short coming of PoW blockchains is their inability to scale to low confirmation times required for typical micro payments. The main reason for this is the unsteady growth in chain weight. This work proposes a simple solution where chain growth is maintained in a steady manner and linear structure of the blockchain is maintained. We also lower confirmation times while maintaining security guarantees.
Links are a better alternative to solutions that try to improve confirmation times by reducing block size and interval.
This work has an Indian patent and has an International PCT filing. It is published in COMSNETS 2022.\\
\textit{URL - https://ieeexplore.ieee.org/abstract/document/9680576}\\
Advisor: Dr Vinay Ribeiro, IIT Bombay

% \item Ongoing Project for "Methods in PoW Blockchain systems for its improved security, stability and performance" with Dr. Vinay Ribeiro, IIT Bombay.  It opens the blockchain to double-spend and selfish mining attacks can lead to forks which exacerbate these attacks while delaying consensus. This also leads to an increased transaction confirmation time and reduced throughput not abling them to scale to low confirmation times which is required for typical micropayments. Naive solutions which reduce block intervals hoping for smoother growth instead induce more forks, negatively affecting security. Other recently proposed schemes ensure smooth chain weight growth, but at the cost of significantly modifying the linear chain structure of Bitcoin. My research tries to prevent forks and reduce confirmation times in existing blockchain systems while maintaining security and preserving a simple chain structure. This project produced two Indian patents. It is also in the application process for International patent filings. Two parallel submissions are also currently under review in top-tier security conferences.


\item \textbf{Blockchain like Relational Database}\\
 This project was in collaboration with IBM IRL, Bangalore during July to October 2017. This project aimed to bring blockchain properties like immutability and decentralization into relational databases to make them more powerful for certain applications in supply chain management.\\
 Advisor: Dr Praveen Jayachandran, IBM IRL

\item \textbf{Near real time consensus using Hashgraph in IoT systems}\\
This project was a part of Indo-Japanese Research projects for communications in IoT Networks during January to July 2017. This project was part of "Work Package 3: Develop an application layer trusted framework" in a series of Indo-japanese collaborations. Here we built a prototype for a lightweight consensus protocol in IoT networks using Hashgarph with BFT consensus as opposed to the heavy PoW consensus.\\
Advisors: Dr Subodh Sharma, IIT Delhi and Dr Kosuke Kaniko, Kyushu University, Japan


% \item \textbf{Computational offload and near real-time consensus for untrustworthy p2p networks}\\
% In this project we are exploring a potential alternative to hash based PoW consensus with useful proof of work.\\
% Advisor: Dr Vinay Ribeiro, IIT Bombay

\end{itemize}
\end{rSection}









\begin{rSection}{Conferences and Workshops}
\begin{itemize}
% \item Invited as speaker at Workshop on Blockchains and Networking at the prestigious ACM Sigmetrics 2022 in June 2022.

\item Conducted the blockchain workshop as part of training for The Central Reserve Police Force (CRPF) on ``Network Security and Cryptography" at IIT Delhi in April 2021.

% \item Invited as Speaker the National Research Evaluation Workshop of the Vishvesvaraya PhD scheme for electronics and IT in Chandigarh in July 2019.

\item Selected to volunteer and organize Comsnets 2019 held in IISc, Bangalore in January 2019. Chaired the workshop on Blockchains as part of the Conference.

\item Invited to the Pepsico Million Women Mentorship Programe in Gurgaon in October 2018.

\item Selected to volunteer at the prestigious ACM Mobicom 2018 held in Delhi in October 2018.

% \item Invited as speaker to the 5th International Workshop on Cyber security in Kyushu University, Fukuoka, Japan in July 2017.

\item Sponsored Participant Grace Hopper Celebration of Women in Computing India (GHCI) 2016 conference held at Bangalore in December 2016.

\item Invited to IBM women in research program called Maitreyee at IBM IRL, Delhi in November 2016.

\item Selected to volunteer at the prestigious VLDB 2016 conference held in Delhi in September 2016.

\item Sponsored to attend the Microsoft Summer school on the Internet of Things in June 2016 held at the Indian Institute of Science, Bangalore.

 \item Volunteered in the 4th International conference on Big Data Analytics in NIT Warangal in December 2015.


\end{itemize}


\end{rSection}



\begin{rSection}{Honors and Awards}
\begin{itemize}
\item A Visvesvaraya PhD fellow availing Government scholarship from July 2015.
\item Sponsored Student of the Indo-Japanese research collaboration by the Indian and Japanese governments. 
\item Qualified UGC-NET National level Exam for Computer Science and Applications in June 2015.
\item Qualified All India Graduate Aptitude Test Engineering(GATE) for Computer Science and Information Technology conducted by MHRD in March 2013 and in March 2015.
\end{itemize}
\end{rSection}

\begin{rSection}{Graduate Teaching Assistant}
\item Networks and System Security [SIL765][Spring 2017, Spring 2018, Spring 2019]
\item Special Topics in High Speed Networks: Blockchain [COL867][Fall 2017]
\item Computer Networks [COL672][Fall 2016, Fall 2018]
\item Introduction to Database Systems [COL632][Spring 2016]
\item Special topics in DB systems: Data Mining [CSL868][Fall 2015]
\end{rSection}
\begin{rSection}{Technical skills}

\begin{tabular}{ @{} >{\bfseries}l @{\hspace{6ex}} l }
Computer Languages &  Python, C/C++, Java, php, SQL \\
Software \& Tools & LaTeX, MS office, ERWIN, Talend \\
Technologies & Bitcoin, Ethereum \\
\end{tabular}

\end{rSection}

\begin{rSection}{Personal Details}

\begin{tabular}{ @{} >{\bfseries}l @{\hspace{6ex}} l }
E-mail &  oviaseshadri@gmail.com \\
Phone & +91 88262 88396 \\
D.O.B. & 2nd July 1991\\
LinkdIn & \href{https://www.linkedin.com/in/ovia-seshadri-14b75124/}{linkedin.com/ovia-seshadri-14b75124/} \\
GitHub & \href{https://github.com/oviaseshadri}{github.com/oviaseshadri}
\end{tabular}

\end{rSection}
\begin{rSection}{References}
% \begin{itemize}
\item 
\textbf{Dr Vinay J. Ribeiro}\hfill {\textbf{Dr Subodh V. Sharma}}\\
Associate Professor\hfill {Associate Professor}\\
vinayr@iitb.ac.in\hfill {svs@cse.iitd.ac.in}\\
Department of Computer Science\hfill {Department of Computer Science}\\
Indian Institute of Technology Bombay (IITB)\hfill {Indian Institute of Technology Delhi (IITD)}

% \item \textbf{Dr Subodh V. Sharma}\\
% Associate Professor\\
% svs@cse.iitd.ac.in\\
% Department of Computer Science\\
% Indian Institute of Technology Delhi (IITD)

\item \textbf{Dr Bijendra N. Jain}\\
Distinguished Professor\\
Bnjain.bits@gmail.com\\
Department of Computer Science\\
Indraprastha Institute of Information Technology(IIIT), Delhi

% \end{itemize}
\end{rSection}

%	EXAMPLE SECTION
%----------------------------------------------------------------------------------------

% \begin{rSection}{Academic Achievements} \itemsep -2pt
% \item Ranked in National Top 0.2\% (amongst 1,200,000 candidates) in JEE Mains 2013 and Top 1\% (amongst 150,000 candidates) in IIT-JEE Advanced 2013
% \item Ranked in the State-wise Top 1\% (amongst 70,000 candidates) in State level Engineering competitive Exam (MP PET)
% \item Stood first in MBD Talent Search Exam conducted by state government, competing against more than 1000 participants  
% \end{rSection}

%----------------------------------------------------------------------------------------
% \begin{rSection}{Relevant Courses}
% \itab{\textbf{Core Courses}} \tab{}  \tab{\textbf{Other Courses}}
% \\ \itab{Fluid Mechanics \& its applications } \tab{}  \tab{Computational Methods in Engineering}
% \\ \itab{Thermodynamics} \tab{}  \tab{Fundamental of Computing} 
% \\ \itab{Heat Transfer \& its applications} \tab{}  \tab{Probability and Statistics} 
% \\ \itab{Mass Transfer \& its applications} \tab{} \tab{Calculus \& Linear Algebra}
% \\ \itab{Transport Phenomena (ongoing)} \tab{} \tab{Introduction to Mechanics}
% % \\ \itab{Process Control (ongoing)} \tab{} \tab{Electrodynamics}

% \end{rSection}

% \begin{rSection}{POSITION OF RESPONSIBILITY}

% \begin{rSubsection}{Techkriti 2015 - Technical and entrepreneurial Festival }{August 2014 - March 2015}{Public Relations}{IIT Kanpur}
% \item Spearheaded a 2-tier team of 40 people to successfully conduct professional shows, exhibitions and talks
% \item Organized talks in Techkriti by eminent personalities like Dr K. Radhakrishnan (Chairman, ISRO), Peter Schultz (Co-inventor, Fibre optics) and David Hilmers (NASA Astronaut) with more than 1000 attendees
% \item Successfully organized Auto expo, Space expo and Defence expo together for the first time in Techkriti
% \item Promoted awareness through social campaigns like Make a wish, Adopt a tree and Teen Suicide Prevention
% \end{rSubsection}

% %------------------------------------------------

% \begin{rSubsection}{Students' Placement Office}{April 2015 - Present}{Internship Coordinator}{IIT Kanpur}
% \item Coordinating with team of 20 students responsible for facilitating internship proceedings of 650 students involving 150 companies
%  \item Responsible for developing contacts with corporate recruitment teams of several firms for internship and placements 
%  \item Organized sessions on Personality Development and Career Awareness by esteemed alumni for over 1600 students
% \end{rSubsection}

% %------------------------------------------------

% \begin{rSubsection}{Hall Executive Committee }{April 2014 - Nov 2014}{Secretary}{IIT Kanpur}
% \item Coordinated with 12 members to led a team of 200 students in inter hall technical, cultural and sports competition of institute 
% \item Planned an annual budget of ₹ 2 lakhs for proper functioning of hostel with more than 400 residents
% \end{rSubsection}

% \end{rSection}

%----------------------------------------------------------------------------------------
% \begin{rSection}{Extra-Cirrucular} \itemsep -3pt
% \item Secured Gold in M.P. State Throw Ball competition and represented district in State Hand Ball competition
% \item Represented Institute in Udghosh’13 and secured second prize in Kho-Kho intramurals
% \item Secured second prize in Dance Drama competition in Galaxy’14, inter hall cultural competition of IIT Kanpur
% \item Won second prize in Electromania, circuit game designing competition in Takneek’13, inter hall technical festival of IIT Kanpur
%  % \item Member, Athletics Team, IIT Kanpur. Attended Summer Sport Camp as a long jumper.
% %\item Trained and disciplined in National Cadet Corps (NCC), IIT Kanpur for a year.
%  %\item  Participated in Vijyoshi Camp 2012 organized at Indian Institute of Science, Bangalore.
%  %\item Won 2nd position in Kho-Kho in Intramurals conducted by Physical Education Section, IIT Kanpur.
%  %\item Pursued French as second language during secondary school from Grade 6 to Grade 10. Also participated in French Song Competition and French G.K. Quiz in Class 10th. %

% \end{rSection}

\end{document}
